\documentclass[12pt]{article}

\usepackage[portuguese]{babel}
\usepackage{indentfirst}
\usepackage{xcolor}

\author{Gabriel da Fonseca Ottoboni Pinho - DRE 119043838}
\title{Laboratório 3 de Computação Concorrente}
\date{09/04/2021}

\begin{document}
\maketitle
\newpage

\section{Tempos}
\subsection{Para $n$ pequeno}
Como esperado,
a versão sequencial é mais rápida que a versão concorrente
para valores de $n$ pequenos.
Aumentar o número de threads aumenta ainda mais a diferença;
o tempo necessário para criar as threads não vale a pena
para valores pequenos de $n$.

\begin{center}
\begin{tabular}{|c|c|c|c|c|}
	\hline
	Versão & $n$ & $t$ & Tempo & Valor calculado para $\pi$\\
	\hline
	Sequencial  & $10^2$ & - & 0.000001s & \colorbox{orange}{3.1}31592903558554\\
	Concorrente & $10^2$ & 1 & 0.000232s & \colorbox{orange}{3.1}31592903558554\\
	Concorrente & $10^2$ & 2 & 0.000247s & \colorbox{orange}{3.1}31592903558553\\
	Concorrente & $10^2$ & 4 & 0.000350s & \colorbox{orange}{3.1}31592903558554\\
	\texttt{M\_PI} & - & - & - & \colorbox{orange}{3.1}41592653589793\\
	\hline
\end{tabular}
\end{center}

\subsection{Para $n$ grande}
Para valores grandes de $n$,
vemos que a versão concorrente consegue
tirar proveito dos threads,
com o tempo sendo cortado pela metade
cada vez que o número de threads dobra.
\begin{center}
\begin{tabular}{|c|c|c|c|c|}
	\hline
	Versão & $n$ & $t$ & Tempo & Valor calculado para $\pi$\\
	\hline
	Sequencial  & $10^9$ & - & 3.572786s & \colorbox{orange}{3.14159265}2588050\\
	Concorrente & $10^9$ & 1 & 3.564133s & \colorbox{orange}{3.14159265}2588050\\
	Concorrente & $10^9$ & 2 & 1.808738s & \colorbox{orange}{3.14159265}2589258\\
	Concorrente & $10^9$ & 4 & 0.918976s & \colorbox{orange}{3.14159265}2589210\\
	\texttt{M\_PI} & - & - & - & \colorbox{orange}{3.14159265}3589793\\
	\hline
\end{tabular}
\end{center}

\section{Conclusão}
As versões sequencial e concorrente
calculam valores levemente diferentes.
Além disso, a versão concorrente calcula valores distintos
dependendo do número de threads.
Essas diferenças podem ser explicadas por
arredondamentos nos cálculos de ponto flutuante.
Nenhuma das duas parece se aproximar de $\pi$
mais rápido que a outra.

\end{document}
